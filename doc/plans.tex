\documentclass{article}[12pt]
\title{Informal Description of Plan for incorporating SNIa into LSST simulations}
%\input{preamble}
\begin{document}
\maketitle
\section{Objectives}
We would like to incorporate supernovae in the LSST catalogs which currently have galaxies. This means that we would like to add information to the existing catalogs, so that at any point, we can query the catalog for the additional light 
from a SN within a particular patch of sky at a particular instant of time. 
This means we have to encode :
\begin{itemize}
\item position: RA, DEC, z of the Supernova. If the SN is associated with a host galaxy, this can be a displacement vector from the host
\item velocity: This can be relative to the host which presumably has its own peculiar velocity 
\item MJD of peak of supernova 
\item rest frame spectral energy distributions (SEDS) of SN
\end{itemize}
with as few variables as possible. 
\section{Possible Science Projects using SNIa in LSST catalogs:} 
Adding SN to the LSST catalogs will enable us to put images of SN 
on the sky observed by LSST, by propagating through the atmosphere and optics 
system. Some studies that this task would enable are:
\begin{enumerate}
\item Obtaining errors on fluxes in light curves from simulations directly 
including effects of difference subtraction. In the language used in DES, this
would be enabling a 'FAKES' only study. Current simulations use survey/observation  characteristics to estimate the errors on measured fluxes and fudge the values to match data from the survey. This clearly requires being able to obtain 
images and do difference imaging.
\item In comparison to current (SNANA like simulations place all SN in a field 
at the center of the field) simulations, this catalog will have more realistic 
large scale structure clustering efffects on SN. This could enable 
studies on the effect of LSST observing strategies and would be a good resource
to use in following up studies like~\cite{Carroll:2014oja}. This could be 
done without processing images, by using error estmates similar to those used in
 SNANA.
\item Further, due to the clustering effect, this tool would be useful in 
trying to study the signature of clustering from SNIa data, using probes
like BAO or the peculiar velocity of SNIa. Clearly, both of these tasks can 
be accomplished much more precisely using (far more numerous) galaxies, and 
therefore it probably is not a useful project. However, a potential advantage 
of using SNIa is that it could provide a distance scale as well. If a good
subset of SNIa host spectra can be obtained, the simultaneous determination of an angular diameter distance from the host sample and the luminosity distance,
 which have to match to a factor of $(1 + z)^2$ could potentially be used to 
study systematic effects.  
\item Being able to study classification of transients, such as separating 
SNIa from AGNs. While SNIa light curves differ sufficiently from AGNo over
long time scales that they are easy to classify, it may be possible to do so 
even on short time scales, enabling better follow-up by relevant groups of interest. Such studies would also require AGN/other transients to be added to the 
simulations.   
\item Being able to study lensing effects of SN studies from simulations,
leading to estimates of how many supernovae are lensed enough for delay times to
be measurable.
\end{enumerate}

\section{Current SN simulations}
Current SN simulations for LSST, DES, SDSS have been done with SNANA. snCOSMO
is another tool that is being built up with such capability, but it currently 
lacks many of the features of SNANA. 

The basic procedure is explained below:
\begin{itemize}
\item Volume at each redshift shell for an observational field is calculated.
\item The number of SN expected to go off during the survey period within that
volume is calculated from an estimate of the SN rate, usually parametrized 
as a power law in redshift of the form 
$\alpha (1 + z )^\beta $ with parameters $\alpha$ and $\beta$ from a suitable 
dataset. These observed rates are supposed to have been corrected for 
incompleteness. 
\item A ``simlib" file is created with entries for each field, with the peak MJD 
distributed uniformly across the survey time, and the position placed at the center of the field. These simulations are not used for any spatial information, 
hence this is mostly OK. A couple of small deficiencies due to this are that 
weather conditions used pertain to the center of the field, rather than the 
actual location of the SN. This also does not have the ability to model overlaps
 in pointings. Of course, these deficiencies are bad if any spatial correlations/lensing properties are to be calculated from the data.
\item SN are simulated on the basis of the simlib file described above, using 
an emperical SN model such as SALT2. In this case, SN SEDS are described 
by 4 parameters a peak MJD parameter, an amplitude, a 'color' parameter $c$, 
and a 'stretch' parameter $x_1$
parameters, so that the distribution approximates the observed distribution 
after selection cuts. The procedure involves trial and error. Once these 
parameters are known, one can use the SALT2 model to obtain a SN SED. 
Finally, a 'color-smearing' term is added according to a prescription. This is a
 random noise term essentially necessary because the SALT2 SED does not describe  the full diversity of SN SEDs. Thus, generation of the SN requires 
a random draw for peak MJD, amplitude, color, x1, and error terms. This neglects
correlations in the properties of SN parameters and the host galaxy parameters,
which are known to exist, but is OK as no host properties are used.
\end{itemize}
\section{Adding SN to LSST catalogs}
In order to add SN to LSST catalogs, we need to be able to do the following :
\subsection{Change Description from Volumetric rate to Galactic Rate}
As described above, current simulations rely on a volumetric rate of SNIa, 
which are then uniformly deposited at the center of a field of view, with the 
number of SNIa determined by the volume based on assumptions of homogeneity. 
We need to associate SN with galaxies. In 
principle, this requires drawing random numbers from the propability 
distribution $P( \theta_{SN} \vert \theta_{gal}).$
For example, we may consider $\theta_{SN} = \{MJD(peak), M_b(peak), x_1, c\}$ and \\
$\theta_{gal}$ = \{morphology, Stellar Mass, Color, star formation rate \}.\\
So,
$$
P(\theta_{SN} \vert \theta_{gal}) = P(\theta_{SN}\vert \theta_{gal}, SN) P(SN \vert \theta_{gal})$$
The first term describes how the properties of SN correlate with the host properties (and is something Eve Kovacs wants to work on too), the second term describes the probability of a galaxy to host a supernova. Now, in order to make contact to the previous simulations described above,
we need (in appropriate units) 
\begin{equation}
\int P(SN \vert \theta_{gal} ) n (\theta_{gal}) d \theta_{gal} = n_SN(z) 
\label{eqn:rate}
\end {equation}
where $n_{SN}(z) $ is the rate of SN, and $n(\theta_{gal})$ is the density of galaxies as a function of their properties in the catalogs. My initial proposal is to go to each redshift bin, use only one galaxy property (stellar mass) 
 and make the term $P(SN\vert \theta_{gal})\propto M$, normalized such that 
the rate is reproduced through Eqn.~\ref{eqn:rate}. 

\subsection{Properties of SNe in catalog:}

For each galaxy, we draw a set of random numbers representing the following 
parameters:
\begin{itemize}
\item SN properties with galaxy type. For SALT2 SNIa these are $\{c, x_1, M_b\}$ 
and the distributions would ultimately depend on the galaxy properties. Right now,
we will not use this. Eve Kovacs (ANL) is planning to study the appropriate 
functions to plug into this. 
\item Peak MJD drawn from a uniform distribution in time, but the normalization 
will be different for different types of galaxies. This is one parameter that 
should probably be the easiest to access. 
\item velocity of the SN with respect to the galaxy, for which we will need three numbers, probably from an isotropic normal distribtion of the same width for all gaalaxies to start with.  This could have something to do with the relative position
of the SN, and also would be useful to use for a transformation of the SED which 
would be interesting for low z (though LSST may not have very low z SN. 
\item position of SN wrt to galaxy position (Probably Center of Mass, or however 
this is recorded in the LSST catalog). This requires three random numbers which 
we can draw from a probability distribution (maybe start with a normal distribution, or if the stellar mass distribution is available, we could leverage that?). This
could also perhaps depend on the velocity of the SN.
\end{itemize} 
%\bibliographystyle{natbib}
\bibliography{plans.bib}
\end{document}

\begin{itemize}
\item obtain SEDs for the simulated supernovae 
SN simulations 
